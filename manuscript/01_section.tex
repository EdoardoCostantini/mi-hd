\section{Introduction}

%\paragraph{(Frame the problem)}

Today’s social, behavioral, and medical scientists have access to large multidimensional data sets that can be
used to investigate the complex roles that social, psychological and biological factors play in 
shaping individual and societal outcomes.
Large social scientific data sets---such as the World Values Survey and the European Values Study (EVS)---are easily accessible to researchers, but making use of the full potential of these data requires dealing with the crucial problem of multivariate
missing data.

Rubin's Multiple Imputation (MI) approach \citep{rubin:1987} was developed to address missing 
responses in surveys.
An MI-based analysis is a three-step process that entails an imputation phase, an analysis phase, and a pooling phase.
The fundamental idea of the imputation phase is to replace each missing data point with $m$ plausible values sampled from 
the posterior predictive distribution of the missing data given the observed data.
This procedure generates $m$ completed versions of the original data that are analyzed separately during the analysis phase, using standard complete data analysis models.
Finally, in the pooling phase, the $m$ estimates of any parameter of interest are pooled following Rubin's rules \citep{rubin:1987} to create the MI parameter estimate.

Since Rubin's seminal work, two main strategies have become popular for multiple imputation of multivariate 
missing data: joint modelling \citep[JM;][ch. 4]{schafer:1997} and full conditional specification (FCS), also known 
as Multiple Imputation by Chained Equations \citep[MICE;][]{vanBuurenEtAl:2006}.
JM relies on defining a joint distribution for the missing data, deriving conditional
distributions for each missing data pattern, and obtaining samples from these distributions by means of a Markov Chain Monte Carlo 
algorithm.
FCS defines conditional distributions for each incomplete variable and performs iterative imputations on 
a variable-by-variable basis.
Compared to the JM approach, FCS can more easily accommodate different distributions of the incomplete variables. It is also easier to preserve unique features in the data, such as interactions between variables or skip patterns in questionnaires when using FCS.

Both the JM and FCS rely on the crucial \emph{missing at random} (MAR) assumption.
Meeting this assumption requires specifying imputation models that include all observed correlates of the missingness.
Omitting a variable that relates to both the missingness and the incomplete variables induces \emph{missing not at random} (MNAR) data.
MI under MNAR can lead to substantial bias in the MI parameter estimates, and 
invalidates hypothesis testing involving the imputed variables. As a result, when choosing which auxiliary variables (i.e., covariates included purely to support the missing data analysis) to include in the imputation model, 
an inclusive strategy (i.e., including many auxiliary variables) is generally preferred to restrictive approach 
(i.e., including few or no auxiliary variables).
An inclusive approach reduces the chances of omitting important correlates of missingness, thereby making the MAR assumption 
more plausible.
Furthermore, the inclusive strategy has been shown to reduce estimation bias and increase efficiency \citep{collinsEtAl:2001} and to
reduce the chances of specifying uncongenial imputation and analysis models \citep{meng:1994}.

Specifying the imputation models for a FCS MI procedure remains one of the most challenging steps in dealing
with missing values for large multidimensional data sets.
In practice, the inclusive strategy faces identification and computational limitations.
One serious risk of an inclusive strategy is the occurrence of singular covariance matrices within the imputation algorithm.
When the predictor matrix is high-dimensional (i.e., the number of recorded units, $n$, is not substantially larger than the number of recorded 
variables, $p$) or afflicted by high collinearity (i.e., some variables are linear combinations of other variables) the covariance matrix of the predictors will be singular.
Singular matrices cannot be inverted, which is a fundamental part of estimating the imputation 
model in most parametric imputation procedures.
As a result, the possible high dimensionality of the predictor matrix resulting from an inclusive strategy 
can prevent a straightforward application of MI or force researchers to make arbitrary choices 
regarding which variables to include in the imputation model.

\subsection{Prior work on high-dimensional imputation}
Recent developments in high-dimensional MI techniques represent interesting opportunities to embrace an
inclusive strategy, without facing its downsides.
High-dimensional Single Imputation methods have been developed in 
an effort to improve the accuracy of the individual imputations \citep[e.g.,][]{kimEtAl:2005, stekhovenBuhlmann:2011, 
d'ambrosioEtAl:2012}. 
However, the main task of social scientists is to make inference about a population based on a sample, and Single Imputation (SI) is inadequate for this purpose because it does not support statistically valid 
inference \citep{rubin:1996}.
The relevant conceptualization of statistical validity was defined by \citet{rubin:1996} as capturing two features of 
estimation.
First, the point estimate of a parameter of interest must be unbiased, and second, the actual 
confidence interval (CI) coverage of the true parameter value must be equal or greater than nominal the
coverage rate.
SI strategies might meet the first requirement but cannot meet the second because they 
do not account for uncertainty regarding the imputation model parameters.
MI, on the other hand, was designed to provide statistically valid inference and, therefore, is 
more suitable for social scientific research.

The combination of MI with high-dimensional prediction models has been directly tackled by algorithms that combine 
FCS with shrinkage methods \citep{zhaoLong:2016, dengEtAl:2016}.
Other researchers have proposed the use of dimensionality reduction to avoid the obstacles of an inclusive strategy.
However, these solutions were either limited to the JM approach \citep{songBelin:2004}, 
or tested exclusively in low-dimensional settings \citep{howardEtAl:2015}.
Finally, tree-based FCS strategies also have the potential to overcome the limitations of inclusive strategies.
The nonparametric nature of decision trees bypasses the identification issues most parametric methods face
in high-dimensional contexts.

\subsection{Scope of the current project}
Including shrinkage methods, Principle Component Analysis (PCA), and decision trees within the FCS framework 
has the potential to simplify the decisions social scientists need to make when dealing with missing values.
However, the lack of comparative research on the performance of these methods makes it difficult for social scientists working with 
large data sets to decide which imputation method to adopt.
In this article, we provide a comparison of the state-of-the-art in high-dimensional imputation algorithms.
We compared seven promising imputation methods in terms of their ability to support statistically valid analyses.
These comparisons were based on three numerical experiments: two Monte Carlo simulation studies and a resampling study using real survey data.

In what follows, we first introduce the missing data treatments that we compared in our study.
Then we present the methodology and results of the three numerical experiments.
We discuss the implications of the results and provide recommendations for applied researchers.
We conclude by describing the limitations of the study and recommending future research directions.
