\maketitle
\section{Resampling Study}

To test the ecological validity of findings in experiment 1 and 2 we have also designed a
resampling study based on European Values Survey data.
By using data gathered for an actual survey, we can mostly observe whether the relative 
performances of the imputation methods change when they are deployed for real data research.
Variables in the EVS data are not generated artificially from continuously normal distributions 
but are discrete numerical items that are treated as such by researchers.

The resampling study follows a similar strategy to that used in the simulations. 
To assess the statistical validity of the different imputation methods we have repeated the 
following steps 1000 times ($R = 1000$):

\begin{itemize}
	\item Data generation - A bootstrap sample $\bm{X}^{*}$ was generated by sampling with replacement $n$ 
		observations from a pre-processed EVS data-matrix. 
		Part of the pre-processing step was some form of imputation used to obtain a pseudo-fully observed 
		input data matrix.
	\item Missing data imposition - Missing values were imposed on a given number of target variables
		in $\bm{X}^{*}$, according to some response model.
	\item Imputations - Each method described in section 2 to deal with missing values was used to impute
		NAs.
	\item Analysis - Two analysis models were fitted to the differently treated data.
		Parameters estimates pooled across the differently imputed datasets for the MI methods and
		stored along with the estimates obtained with single imputation methods and complete case 
		analysis.
\end{itemize}

	The average estimate, over the $R$ repetitions, obtained with the Gold Standard approach are considered 
	as "true" reference values of the parameters in the analysis models.
	The $R$ estimates obtained with all other methods are used to obtain performance measures for each imputation 
	method using the same criteria described for study 1 and 2 (see \ref{criteria}).

	The code to Run the simulation was written in the R statistical programming language (version 4.0.3). 
	The resampling study was run using a 2.6 GHz Intel Xeon(R) Gold 6126 processor, 523780 MB of Memory. The
	operating system was Windows Server 2012 R2.

	Computations were run in parallel across the available cores (between 30). Parallel computing 
	was implemented using the R package 'parallel' and to ensure replicability of the findings seeds were
	set using the method by \cite{lecuyer:2002} implemented in the R package 'rlecuyer'

\subsection{Methods}

\paragraph{Data preparation}
	EVS is a standardizes cross-sectional survey with a representative sample of more than 60,000 
	people, across more than 30 countries, interviewed via Web, post or face-to-face.
	For this study we have used the third pre-release of the 2017 wave of EVS data \citep{EVS:2017}.
	The original dataset contained 55,000 observations in 34 countries.

	We selected only the four European Founding Countries included in the data (France, Germany,
	Italy, and the Netherlands) and excluded all columns of the data that were either duplicated
	information (recoded versions of other variables), or linked to meta data (e.g. time of interview,
	mode of data collection). 
	All missing values were filled in with a run of a single imputation predictive mean matching (PMM) 
	which allowed us to obtain a pseudo fully-observed dataset. PMM was chosen for the task as it 
	is an effective, flexible imputation method that maintains the distributional characteristics of 
	the original data. 
	The full cleaning process is more systematically described in the appendix.

	At the end of this data cleaning process, we ended up with a fully-observed dataset
	of 8045 observations ($n$), across 4 countries, and 243 variables ($p$).

\paragraph{Analysis model(s)}
	Describe model 1: effect of dimensions trust on euthanasia acceptance

	Describe model 2: effect of gender on left/right voting behaviour

\paragraph{Missing data imposition}

	Missing data were imposed on 6 variables according to the same strategy as in \ref{subsub_missing}.
	The target variables we identified were the two dependent variables in models ... and ... : 
	tendency to vote left or right, and attitudes toward euthanasia.
	We imposed missing values on predictors in the linear models as well. In particular, religiosity and the
	three items making up the nativist attitudes scale.

	The response model form is the same as in \label{eqn:rm} and 3 variables were included in $\tilde{X}$: 
	age, education, and an item measuring trust in new people. These are plausible variable that influence
	response tendencies in participants: older people usually have higher item non-response rates than younger;
	so do lower educated compared to higher educated people; we have also assumed that people that trust less
	new people will tend to withhold more information from the interviewer.

\paragraph{Conditions}
	There were only two conditions for the resampling study: low and high dimensional imputation.
	As the number of predictors in the data is fixed ($p = 250$), the dimensionality of the data is
	changed by defining different sizes for the sample taken from the pseudo-fully observed data.
	We chose only two values for $n$, namely $1000$ and $300$, corresponding to the low and high 
	dimensional condition.

\subsection{Results}
\paragraph{Bias}
	Report results of comparison in terms of estimates bias for relevant parameters.

	Given concise idea of implications. Avoid higher level comparisons.

\paragraph{Confidence Interval Coverage}
	Report results of comparison in terms of confidence interval coverage of the "true" values of parameters 

	Given concise idea of implications. Avoid higher level comparisons.
