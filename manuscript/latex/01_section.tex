\section{Introduction}

%\paragraph{(Frame the problem)}

Today’s social, behavioral and medical scientists are blessed with a wealth of large, high-quality data that
can help investigate the complex relationships between social, psychological and biological factors in 
shaping individual and societal outcomes.
Large social scientific datasets, such as the World Values Survey (WVS), or the European Values Study (EVS), 
are easily available and initiatives have been undertaken to link and extend these datasets into a full systems 
of linked open data (LOD).

Making use of the full potential of these data sets requires dealing with the crucial problem of multivariate
missing data.
Rubin's Multiple Imputation approach \citep{rubin:1987} was developed to specifically address the issue of missing 
responses in surveys.
The basic idea underlying MI is to repeatedly sample replacement values for each missing data point, by sampling from a 
predictive distribution given observed data.
This procedure leads to the definition of multiple datasets, each imputed with different samples from the predictive
distribution, that can be analyzed separately using standard complete-data models.
Results can then be pooled following Rubin's rules \citep{rubin:1987}.

Multiple Imputation relies on the crucial Missing At Random (MAR) assumption.
Meeting this assumption requires specifying imputation models for the MI procedure that include all observed
variables that are correlates of missingness.
Omitting from the imputation models an observed predictor related to missingness might lead to substantial 
bias in the estimation of any analysis model, and invalidate hypothesis testing involving the imputed variables.

As a result, when it comes to defining the set of auxiliary variables for the imputation models within an MI procedure, 
an inclusive strategy (i.e. including numerous auxiliary variables) is generally preferred:
compared to restrictive approach (i.e. including few or no auxiliary variables), it reduces the chances of 
omitting important correlates of missingness, making the MAR assumption more plausible.
Furthermore, the inclusive strategy has been shown to reduce estimation bias and increase efficiency \citep{collinsEtAl:2001},
as well as reducing the chances of specifying uncongenial imputation and analysis models \citep{meng:1994}.

In practice, however, an inclusive strategy increases the dimensionality of the imputation models 
and identification and computational limitations often force researchers to make arbitrary decisions on what 
predictors to include in the imputation models.
One serious risk of an inclusive strategy is the occurrence of singular matrices within the imputation algorithm.
When data is high-dimensional ($n$ is \emph{not} substantially larger than $p$) or afflicted by high collinearity 
(correlation among certain variables is so high that some of their linear combinations have no variance) the data 
covariance matrix is singular. 
Singular matrices are not invertible, an operation that is fundamental in the estimation of the imputation 
models in any parametric Multiple Imputation procedure.
As a result, the possible high dimensionality of the observed data matrix, resulting from an inclusive strategy, 
can prevent a straightforward application of MI algorithms, such as MICE \citep{vanBuuren:2012}, or force
researcher to make difficult choices regarding which variables to use.

\paragraph{Background}
Recent developments in high dimensional multiple imputation techniques represent interesting 
opportunities to embrace an inclusive strategy without facing the risk of including too many superfluous 
auxiliary variables in the imputation models. 
Some researchers focused on high-dimensional single imputation methods in an effort to improve the 
accuracy of individual imputations \citep{kimEtAl:2005, stekhovenBuhlmann:2011, d'ambrosioEtAl:2012}. 
However, the main task of social scientists is to make inference about a population based on a sample of observed 
data points, and single imputation is simply inadequate for this purpose: it does not guarantee unbiased and confidence 
valid estimates of the parameters of interest \citep{rubin:1996}.

Multiple Imputation is more suitable for the type of research social scientists are involved in. 
Its combination with high dimensional prediction models has been directly tackled by specific algorithms combining MICE 
with shrinkage methods \citep{zhaoLong:2016, dengEtAl:2016}, dimensionality reduction methods \citep{songBelin:2004, 
howardEtAl:2015}, and even non-parametric prediction trees \citep{reiter:2005, burgetteReiter:2010, dooveEtAl:2014, 
shahEtAl:2014}.

Although some of these approaches have been tested in proper high-dimensional contexts, many of them have 
been either proposed or tested exclusively for low-dimensional imputation settings.
These methods have the potential of simplifying the decisions social scientists need to make when dealing 
with missing values, but so far there has been little research on their comparative performances when applied 
to data these researchers actually want to use.

\paragraph{Scope}
With this article, we set out to provide a comparison of state-of-the-art high-dimensional imputation algorithms 
that do not require the researcher to make decisions on which variables to include in the procedure.
We compared imputation methods based on their ability to allow inferential statements that are as statistically 
valid as if they were made on a dataset without missing data.
Hence, in assessing the methods performances, the primary focus of this article was the \emph{statistical validity}
\citep{rubin:1996} of the substantive analysis performed on data treated with different high-dimensional 
MI procedures.
The comparison was developed through two simulation studies and a resampling study using real survey data.

\paragraph{Outline}
This paper is organized as follows. 
Section 2 discusses the imputation methods compared.
Section 3 presents the two simulation studies, their design and results.
Section 4 presents the resampling study performed on real survey data.
Section 5 discusses the implication of the combined results of the simulation and resampling studies.
Finally, section 6 provides concluding remarks, a description of the limitations of the study, and  
future research directions we want to take.
