\maketitle
\section{Introduction}

\paragraph{(Frame the problem)}

Today’s social and behavioral scientists are blessed with a wealth of large, high-quality and publicly available 
social scientific datasets such as the Longitudinal Internet Studies for the Social Sciences (LISS) Panel and 
the European Values Study (EVS), with initiatives being undertaken to link and extend these datasets into a full
system of linked open data (LOD).
Making use of the full potential of these data sets requires dealing with the crucial problem of multivariate
missing data.

The tools researchers working with these data sets need to correct for the bias introduced by nonresponses 
require special attention.
In general, when performing Multiple Imputation, data handlers tend to prefer including 
more, rather than less, predictors in the imputation models. This reduces the chances of uncongenial imputation 
and analysis models \citep{meng:1994} and of leaving out important predictors of missingness.
On top of this standard source of dimensionality, the large number of items recorded in surveys, coupled with their
longitudinal nature, and the necessity of preserving complex interactions and nonlinear relations, easily produces 
high-dimensional ($p>n$) imputation problems.

When data is sparse ($n$ not substantially larger than $p$) or afflicted by high collinearity (correlation among 
certain variables is so high that some of their linear combinations have no variance) the data covariance matrix
is singular. Singular matrices are not invertible, an operation that is fundamental in the
estimation of imputation models in any parametric Multiple Imputation procedure.
As a result, high dimensionality of the data matrix prevents a straightforward application of imputation algorithms, 
such as MICE \citep{vanBuuren:2012}.

High-dimensional data imputation settings represent both an obstacle and an opportunity in this sense: an 
obstacle, as in the presence of high-dimensional data it is simply not possible to include all available variables 
in standard parametric imputation models; 
an opportunity, because the large amount of features available has the potential to reduces the chances of 
leaving out of the imputation models important predictors of missignenss.

\paragraph{(Discuss background literature)}
Many solutions have been proposed to deal with missing values in high dimensional contexts. Some researchers
have focused on single imputations in an effort to improve the accuracy of individual imputations \citep{kimEtAl:2005, 
stekhovenBuhlmann:2011, d'ambrosioEtAl:2012}. 
However, the main task of social scientists is to make inference about a population based on a sample of observed 
data and single imputation is simply inadequate for this purpose: it does not guarantee unbiased and confidence 
valid estimates of the parameters of interest \citep{rubin:1996}.

Multiple Imputation is more suitable for the task. Its application to high dimensional data has been directly tackled 
by specific algorithms using either shrinkage or dimensionality reduction methods
\citep{songBelin:2004, zhaoLong:2016, dengEtAl:2016}. 
Furthermore, there are other methods, that could potentially suit well the purpose, but have been tested only in 
low-dimensional settings \citep{burgetteReiter:2010, dooveEtAl:2014, howardEtAl:2015}.

\paragraph{(Focus/Reason to write paper)}
With this article we set out to provide a comparison of these state-of-the-art imputation algorithms in 
high-dimensional scenarios. We compare methods based on their ability to allow inferential statements that 
are as valid as if they were made on a dataset without missing data.
The comparison is developed both through simulation studies and a real survey data application.

\paragraph{(Content Summary)}
This paper is organized as follows. 
Section 2 discusses the imputation methods compared.
Section 3 presents two simulation studies, their design and the result of the comparison.
Section 4 presents a resampling study performed on the 2017 wave of the EVS.
Section 5 discusses the implication of the combined results of the simulation and resampling studies.
Finally, section 6 provides concluding remarks, description of the limitations of the study, and  
future directions we want to take.
