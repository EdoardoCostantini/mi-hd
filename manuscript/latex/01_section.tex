\section{Introduction}

\paragraph{(Frame the problem)}

Today’s social, behavioral and medical scientists are blessed with a wealth of large, high-quality data that
can help investigate the complex relationships between social, psychological and biological factors in 
shaping individual and societal outcomes.
Large social scientific datasets, such as the European Values Study (EVS), Longitudinal Internet Studies for the 
Social Sciences (LISS Panel), are easily available and initiatives have been undertaken to link and extend these 
datasets into a full systems of linked open data (LOD).
Furthermore, there are many ways linking different sources of data can be advantageous \citep{jutteEtAl:2011}, from 
linking survey/administrative data with case-control studies to investigate the effects of socio-economic factors in 
shaping health outcomes \citep{kozyrskyjEtAl:2009} [NEEDS BETTER CITATION]; to life-course and trans-generational studies 
monitoring social, psychological and medical features [NEEDS CITATION]

Making use of the full potential of these data sets requires dealing with the crucial problem of multivariate
missing data.
Missing data can occur on these types of data because of traditional reasons (e.g. attrition, unwillingness
to answer sensitive questions), because of errors in the linkage, or because some individuals do not interact
with a specific service/activity recorded in the considered data \citep{harronEtAl:2017}.

The tools researchers working large social surveys and linked data need to correct for the bias introduced by non-responses 
require special attention.
In general, when performing Multiple Imputation (MI) \citep{rubin:1987}, one of the most widespread principled method to deal 
with missing cases, data handlers tend to prefer including more, rather than less, predictors in their imputation models.
This practice increases the dimensionality of the imputation models but reduces the chances of specifying uncongenial imputation and 
analysis models \citep{meng:1994} and of leaving out important predictors of missingness, which is important to meet the MAR 
assumption, a basic requirement for proper imputations.
On top of this standard source of dimensionality, the large number of items included in survey and linked data, coupled 
with their longitudinal nature, and the necessity of preserving complex interactions and nonlinear relations, easily 
produces high-dimensional ($p>n$) imputation problems.

When data is sparse ($n$ is \emph{not} substantially larger than $p$) or afflicted by high collinearity (correlation among 
certain variables is so high that some of their linear combinations have no variance) the data covariance matrix
is singular. 
Singular matrices are not invertible, an operation that is fundamental in the estimation of imputation models in 
any parametric Multiple Imputation procedure.
As a result, high dimensionality of the data matrix prevents a straightforward application of MI algorithms, 
such as MICE \citep{vanBuuren:2012}.

However, high-dimensional data imputation settings represent both an obstacle and an opportunity: an 
obstacle, as in the presence of high-dimensional data it is simply not possible to include all available variables 
in standard parametric imputation models; 
an opportunity, because the large amount of features available has the potential to reduces the chances of 
leaving out of the imputation models important predictors of missingness.

\paragraph{(Discuss background literature)}
Many solutions have been proposed to deal with missing values in high dimensional contexts. Some researchers
have focused on single imputations in an effort to improve the accuracy of individual imputations \citep{kimEtAl:2005, 
stekhovenBuhlmann:2011, d'ambrosioEtAl:2012}. 
However, the main task of social scientists is to make inference about a population based on a sample of observed 
data and single imputation is simply inadequate for this purpose: it does not guarantee unbiased and confidence 
valid estimates of the parameters of interest \citep{rubin:1996}.

Multiple Imputation is more suitable for the task. Its application to high dimensional data has been directly tackled 
by specific algorithms using either shrinkage or dimensionality reduction methods
\citep{songBelin:2004, zhaoLong:2016, dengEtAl:2016}. 
Furthermore, other methods, that could potentially suit well the purpose, are the use of dimensionality reduction
within the imputation models \citep{howardEtAl:2015}, or the use of non-parametric prediction trees 
\citep{burgetteReiter:2010, dooveEtAl:2014}.
However, most of these have been either proposed or tested exclusively in low-dimensional imputation 
settings.

\paragraph{(Focus/Reason to write paper)}
With this article we set out to provide a comparison of these state-of-the-art imputation algorithms in 
high-dimensional scenarios. 
We compare imputation methods based on their ability to allow inferential statements that are as valid as 
if they were made on a dataset without missing data.
Hence, in assessing the methods performances, the primary focus of this article is the \emph{statistical validity}
\citep{rubin:1996} of the substantive analysis performed on data treated with different high-dimensional 
MI procedures.
The comparison is developed both through simulation studies and a real survey data application.

The end goal is to provide recommendations for applied researchers in how do deal with missing values in
a principled and achievable manner when faced with large social surveys and linked data.

\paragraph{(Content Summary)}
This paper is organized as follows. 
Section 2 discusses the imputation methods compared.
Section 3 presents two simulation studies, their design and the result of the comparison.
Section 4 presents a resampling study performed on the 2017 wave of the EVS.
Section 5 discusses the implication of the combined results of the simulation and resampling studies.
Finally, section 6 provides concluding remarks, description of the limitations of the study, and  
future directions we want to take.
