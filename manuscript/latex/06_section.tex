\section{Discussion}

	The inclusion of State-of-the-art modern regression techniques within the MICE framework has the potential
	to simplify the use of MI for social scientist.
	We studied bias and coverage of parameter estimates after imputation with 7 high-dimensional data 
	imputation methods.
	Although extensive simulation studies had already been carried out by the researchers proposing these methods, 
	no comparison study had been developed to assess their relative performances.
	Our research fills this gap and provides initial insights into applying such methods in social scientific research.
	In this section, we discuss the overall performance of the methods and we give recommendations for social scientists 
	facing high-dimensional data imputation problems.

\paragraph{Methods that do not work well}

	We found that bridge is inadequate to deal with  high-dimensional data imputation problems.
	In both the simulation and resampling study the use of a fixed ridge penalty within the imputation
	algorithm manifested the same undesirable performance.
	The method worked well when many predictors were included in the imputation model, but the imputation
	task remained low dimensional.
	However, bridge led to extreme bias and unacceptable confidence interval coverage in all the high 
	dimensional conditions.

	MissForest, the high-dimensional data SI method, leads to low estimation bias.
	However, it results in severe confidence interval under-coverage of the true parameter 
	values.
	Under-coverage coupled with unbiased estimates indicates that too little uncertainty is incorporated in 
	the imputation procedure, which is to be expected from a single imputation approach.
	As a result, missForest should be avoided by a social scientist with the goal of drawing inferential 
	conclusions from their data analysis.

\paragraph{Methods that work best}

	IURR and MI-PCA were the two strongest performers.
	IURR excelled with the smallest estimation bias for item means, variances and regression coefficients.
	The method also produced the small deviations from nominal coverage rates for these parameters.
	Furthermore, the negative covariance estimation bias introduced by IURR in the high-dim-high-pm conditions
	only slightly exceeded the 10\% threshold and the CI coverage was just around 0.9.
	Comparatively, most of the other MI methods resulted in covariance PRBs larger than 20\%, and CICs 
	well below 0.9.

	IURR is easy to specify.
	Compared to regular low dimensional MI, IURR does not require the imputer to make choices regarding which variables
	are relevant for the imputation procedure, and the only additional decision required of the imputer is the 
	number of folds for the cross-validation of lasso penalties.
	As a result, IURR is an extremely appealing method for large surveys imputation.
	However, IURR is a relatively computationally intensive.
	If the number of variables with missing values is large, IURR might result in prohibitive imputation time.
	In such a scenario, a researcher might prefer to address imputation with the MI-PCA method.

	MI-PCA showed low bias and good coverage for both item means and covariances in experiments 1 and 2.
	Although it exhibited large bias of the item variances, the relationships between variables with missing values 
	were always correctly estimated.
	It was the only method resulting in low bias and close-to-nominal CI coverage of the true covariance values,
	even in the high-dimensional conditions.
	Furthermore, it produced the lowest bias for the latent factor loadings.	
	MI-PCA also resulted in low bias and CIC close to nominal rates for the focal regression coefficient
	in Experiment 3.
	Finally, when the CICs obtained with MI-PCA deviated significantly from nominal rates, they over-covered.
	This tendency is less worrisome than under-coverage as it leads to conservative, rather than liberal, 
	inferential conclusions.
	
	Compared to regular low dimensional MI, using MI-PCA requires to make decisions only on the number of 
	principal components to extract.
	As a result, this method is excellent approach for data analysts interested in testing theories on 
	large social scientific datasets with missing values.

\paragraph{Methods with mixed results}
	DURR produced low bias and good CI coverage for item means, variances and regression coefficients.
	However, compared to IURR, it suffered from greater performance deterioration when applied to 
	high-dimensional data. 
	As a result, DURR should not be preferred to IURR.

	The tree-based MI methods, MI-CART and MI-RF, produced large covariance bias in experiments 1 and 2.
	Although, bias for means, variances, and regression coefficients was acceptable, it was usually larger 
	than that obtained by all other MI methods.
	In terms of CI coverage, they showed significant large under-coverage of most parameters in 
	the high-dim-high-pm conditions.

	There was little difference in performances between the use of CART and Random Forests 
	as building blocks of the imputation algorithm. 
	When a difference was noticeable, it was in favor of the use of the simpler single CART,
	which is in line with what \cite{dooveEtAl:2014} found.
	Although the non-parametric nature of these approaches elegantly avoids imputation model over-parametrization,
	these methods are outperformed by IURR and MI-PCA.

	Blasso resulted in low item means and variances bias, even in the high-dimensional conditions.
	While the covariance bias was large in experiments 1 and 2, blasso performed well in the resampling study, 
	where the overall biasing performance was similar to that of MI-OP.
	In terms of confidence interval coverage, blasso showed poor performances resulting in either CI 
	under-coverage or CI over-coverage of true parameter values in almost all high-dimensional conditions, 
	across the three different experimental set ups.
	Furthermore, blasso did not fair particularly well in the recovery of the latent structure in our second 
	experiment.
	Its factor loading PRBs were the highest among the MI methods.

	Theses mixed performances of blasso are also accompanied by a few obstacles to its application for social 
	scientific research.
	Using \cite{hans:2010}'s Bayesian Lasso requires the specification of 6 hyper-parameters, which 
	introduces more researcher degrees of freedom and demands a strong grasp of Bayesian statistics.
	Furthermore, the method has not currently been developed for multi-categorical data imputation,
	a common task in the social sciences.
	As a result, blasso is not recommended for imputation of large social scientific datasets.
