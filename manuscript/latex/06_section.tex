\section{Conclusions}

\paragraph{Recommendations / Take-home message}
	[UNFINISHED] Give recommendations / take-home message in one or two paragraphs

\paragraph{Limitations and future directions}

	As this work aimed at comparing current implementations of different methods, some limitations
	to the scope of the simulation and resampling studies were imposed by the current state of development of 
	the different methods.
	For example, both IURR/DURR and MI-PCA allow imputation of data with any distribution:
	IURR and DURR have already been discussed for categorical data imputation,
	and MI-PCA can be performed with any standard imputation model for categorical data.
	Blasso has not been formally developed for multi-categorical imputation target variables yet, which limited 
	the study to work with missing values on variables that are either continuous in nature or usually considered
	as such in practice.
	Furthermore, the interesting inclusion of interactions and squared terms in the imputation models was
	not explored as their inclusion as not been developed to the same extent across the different methods 
	
	[UNFINISHED] The resampling study compared results only for two analysis models and showed some variation in
	which methods were top performer.

\iffalse 
	Some promising methods were excluded because their current implementation did not allow to meet
	the goals of this study.
	For example, BART is currently implemented to impute the covariates for a given analysis model.
	It assumes that the dependent variable is fully observed which is something that does not fit well
	with the application on social surveys where the dependent variable is as likely as any other to be 
	afflicted by missingness.
	Furthermore, it's reliance on the definition of an analsyis model before the imputation procedure can 
	be run is considered by the authors of this paper as an undesirable feature: in general, for imputation
	of social surveys there is a preference for imputation models that can be suggested independently of
	the type and formulation of the analysis model.
\fi
