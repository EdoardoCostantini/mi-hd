\maketitle
\section{Discussion}

Some of the most popular solutions currently implemented in the R package 'mice' to deal with high dimensionality of 
the data (i.e. bridge and MI-RANF) proved to be quite unsatisfactory in dealing with high-dimensional imputations compared 
to the other approaches considered. In particular, IURR and MI-PCA result to be clear winners, with Blasso being a 
worthy challenger.

When deciding which high-dimensional imputation approach to employ, the type of statistics a researcher cares most about should be taken 
into consideration. Indeed, while IURR, MI-PCA, and Blasso show overall the best performances, they revealed some statistics-
specific weaknesses: in the most challenging condition, IURR and Blasso showed substantially biased covariances, 
while MI-PCA showed poor performance in terms of bias (and coverage) for the variances.

Unreported results from the analysis of linear regressions, involving the variables with imputed values, confirms
the importance of considering the type of analysis and statistics, as blasso was even more competitive in the recovery 
of regression coefficients.

Confidence intervals coverage analysis lead to the conclusion that the reduction in confidence 
validity for all the high-dimensional imputation methods is a function of the proportion of missing cases, not the 
dimensionality of the data. While this was perhaps to be expected, it clearly shows the usefulness of high-dimensional 
imputation methods.  
Of particular interest was that MI-PCA outperformed all other methods showing consistently good
coverage even in the conditions with a high proportion of missing values.

This simulation study is but one part of a larger endeavour that includes other simulation experiments that 
monitor the performances of high dimensional imputation methods analysed as a latent structure is added to the data 
generation mechanism, and as interactions come into play within the analysis model and the missing data imposition.

