\section{Discussion}

	The inclusion of high-dimensional prediction techniques within the MICE framework has the potential
	to simplify the use of MI for social scientists.
	We studied the bias and CI coverage of parameter estimates after imputation by seven high-dimensional MI methods.
	Although extensive simulation studies had already been carried out by the researchers proposing these methods, no comparison study had been developed to assess their relative performance.
	Our research fills this gap and provides initial insights into applying such methods in social scientific research.
	In this section, we discuss the overall performance of the methods and we give recommendations for social scientists 
	facing high-dimensional imputation problems.

\subsection{Methods that do not work well}

	We found that bridge is an inadequate solution to high-dimensional data imputation problems.
	In both the simulation and the resampling study, the use of a fixed ridge penalty within the imputation
	algorithm manifested the same undesirable performance.
	The method worked well when the imputation
	task remained low-dimensional.
	However, bridge led to extreme bias and unacceptable CI coverage in nearly all the high-dimensional conditions.

	We have also confirmed that missFor---the only single imputation method we evaluated---leads to low estimation 
	bias but results in severe CI under-coverage.
	Under-coverage coupled with unbiased estimates indicates that too little uncertainty is incorporated into 
	the imputation procedure, which is to be expected from a single imputation approach.
	As a result, missFor should be avoided by social scientists who wish to draw inferential 
	conclusions from their data analysis.

\subsection{Methods that work well}

	IURR and MI-PCA were the two strongest performers.
	IURR produced the smallest estimation bias for item means, variances and regression coefficients.
	The method also produced small deviations from nominal coverage rates for these parameters.
	Furthermore, the large covariance estimation bias introduced by IURR in the high-dim-high-pm conditions
	only slightly exceeded the $|\text{PRB}| = 10$ threshold and the CIC was around 0.9.
	Comparatively, most of the other MI methods resulted in covariance $|\text{PRB}|$s larger than 20, and CICs 
	well below 0.9.

	IURR does not require the imputer to make choices regarding which variables
	are relevant for the imputation procedure.
	The only additional decision required of the imputer is selecting the number of folds to use when cross-validating the penalty parameter.
	As a result, an IURR imputation run is easy to specify, which makes IURR an appealing method for imputation of large social scientific 
	datasets.
	However, IURR is relatively computationally intensive.
	If the number of variables with missing values is large, IURR might result in prohibitive imputation time.
	In such a scenario, a researcher might prefer to address imputation with the MI-PCA method.

	MI-PCA showed low bias and good coverage for both item means and covariances in experiments 1 and 2.
	Although it exhibited a large bias of the item variances, the relationships between variables with missing values 
	were always correctly estimated.
	It was the only method resulting in low bias and close-to-nominal CI coverage of the true covariance values,
	even in the high-dimensional conditions.
	%Furthermore, it produced the lowest bias for the latent factor loadings.	
	%MI-PCA also resulted in low bias and CIC close to nominal rates for the focal regression coefficient in Experiment 3.
	Finally, when the CICs obtained with MI-PCA deviated significantly from nominal rates, they over-covered.
	In most situations, this tendency is less worrisome than under-coverage as it leads to conservative, rather than liberal, 
	inferential conclusions.
	Our results suggest that MI-PCA is a promising approach for data analysts interested in testing theories on 
	large social scientific datasets with missing values. We are currently conducting research to explore---and extend---the capabilities of the MI-PCA approach more fully.

\subsection{Methods with mixed results}
	DURR produced low bias and good CI coverage for item means, variances, and regression coefficients.
	However, compared to IURR, it suffered from greater performance deterioration when applied to 
	high-dimensional data. 
	As a result, our results suggest that DURR should not be preferred to IURR.

	There was little difference in performance between the use of CART and random forests as elementary imputation methods within the MICE algorithm.
	In line with what \cite{dooveEtAl:2014} found, when a difference was noticeable, the simpler CART generally outperformed the more complex random forests.
	Both MI-CART and MI-RF produced large covariance bias in experiments 1 and 2.
	Although bias for means, variances, and regression coefficients was acceptable, it was usually larger 
	than that obtained by other MI methods.
	Furthermore, in terms of CI coverage, both methods showed large under-coverage of most parameters in 
	the high-dim-high-pm conditions.
	Although the nonparametric nature of these approaches elegantly avoids over-parameterization of imputation models,
	these methods were still outperformed by IURR and MI-PCA.

	Blasso resulted in low biases for item means and variances, even in the high-dimensional conditions.
	While the covariance bias was large in experiments 1 and 2, blasso performed well in the resampling study, 
	where the overall bias levels were similar to those of MI-OP.
	In terms of CI coverage, blasso showed poor performance resulting in either CI 
	under-coverage or CI over-coverage in almost all high-dimensional conditions.
	%Furthermore, blasso did not fare particularly well in the recovery of the latent structure in our second experiment.
	%Its factor loading biases were the highest among the MI methods.

	The mixed performance of blasso is also accompanied by a few obstacles to its application for social 
	scientific research.
	Using \cite{hans:2010}'s Bayesian Lasso requires the specification of six hyper-parameters, which 
	introduces more researcher degrees of freedom and demands a strong grasp of Bayesian statistics.
	Furthermore, the method has not currently been developed for multi-categorical data imputation,
	a common task in the social sciences.
	As a result, we do not recommend blasso for imputation of large social scientific datasets.
